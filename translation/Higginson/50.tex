Whatever rules you have adopted, abide by them  as laws, and as if you would be
impious to  transgress them;  and do not  regard what anyone  says of  you, for
this, after  all, is no  concern of  yours. How long,  then, will you  delay to
demand of yourself  the noblest improvements, and in no  instance to transgress
the  judgments of  reason? You  have received  the philosophic  principles with
which you ought to  be conversant; and you have been  conversant with them. For
what  other  master,  then,  do  you  wait as  an  excuse  for  this  delay  in
self-reformation?  You are  no longer  a boy  but a  grown man.  If, therefore,
you  will  be  negligent  and  slothful,  and  always  add  procrastination  to
procrastination, purpose  to purpose, and fix  day after day in  which you will
attend to  yourself, you  will insensibly continue  to accomplish  nothing and,
living and  dying, remain of  vulgar mind.  This instant, then,  think yourself
worthy of living as a man grown up and a proficient. Let whatever appears to be
the best be to you an inviolable law.  And if any instance of pain or pleasure,
glory or disgrace, be set before you,  remember that now is the combat, now the
Olympiad comes on,  nor can it be put  off; and that by one  failure and defeat
honor may  be lost or won.  Thus Socrates became perfect,  improving himself by
everything, following reason alone. And though  you are not yet a Socrates, you
ought, however, to live as one seeking to be a Socrates.
