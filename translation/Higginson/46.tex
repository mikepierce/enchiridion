Never proclaim  yourself a philosopher, nor  make much talk among  the ignorant
about your principles, but show them  by actions. Thus, at an entertainment, do
not discourse how people ought to eat,  but eat as you ought. For remember that
thus Socrates also  universally avoided all ostentation. And  when persons came
to him and  desired to be introduced  by him to philosophers, he  took them and
introduced them; so well did he bear  being overlooked. So if ever there should
be  among the  ignorant any  discussion  of principles,  be for  the most  part
silent. For there  is great danger in hastily throwing  out what is undigested.
And if anyone tells  you that you know nothing, and you are  not nettled at it,
then you may  be sure that you have  really entered on your work.  For sheep do
not hastily throw up the grass to  show the shepherds how much they have eaten,
but, inwardly digesting their food, they produce it outwardly in wool and milk.
Thus, therefore,  do you  not make  an exhibition before  the ignorant  of your
principles, but of the actions to which their digestion gives rise.
